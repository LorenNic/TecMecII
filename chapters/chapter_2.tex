\chapter{Lavorazioni per asportazione di truciolo}\label{chp:AsportaTruciolo}
Il principio di funzionamento delle lavorazioni per asportazione di truciolo è
differente da quanto visto fin ora.
Nelle lavorazioni per deformazione plastica si avevano delle deformazioni lente.
\emph{"Per cui il materiale aveva il tempo di adattarsi"}.
Per l'asportazione di truciolo così non può essere. Infatti si suppone che il
materiale venga rotto, tra l'altro il più rigidamente possibile.
Dunque le deformazioni richieste devono essere a più alta velocità.
Ciò però crea delle problematiche non indifferenti:
\begin{itemize}
\item deformazioni veloci non sono rappresentabili tramite prove classiche.
\item Si rende necessario idealizzare il processo e man mano aggiungere ipotesi più realistiche.
\end{itemize}

\section{Introduzione}
Nelle lavorazioni viste fin ora si portava il materiale a deformazione, anche per la tranciatura
nel momento in cui il materiale subiva una deformazione critica fino all'innesco di cricche da cui poi
avveniva la separazione del materiale.
Per l'asportazione del truciolo non può essere così: la deformazione avviene in tempi molto più rapidi,
per cui il materiale ha comportamenti differenti da quelli visti in precedenza.
Inoltre, considerando il materiale di scarto, per le lavorazioni a deformazione si può recuperare ed 
eventualmente riciclare.
Per le lavorazioni ad asportazione la cosa è molto più complicata e difficile.
Il valore del truciolo è pressoché nullo e il suo recupero può essere complicato 
per via delle molteplici varietà di materiale che possono subire tali lavorazioni.
Alla tabella \ref{examp:VantSvant} sono riportati alcuni vantaggi e svantaggi di tali lavorazioni.

\begin{example}{Vantaggi e Svantaggi}
\centering
\begin{tabularx}{\textwidth}{XX}
\toprule
\textbf{Vantaggi} & \textbf{Svantaggi}\\
\midrule
Si possono ottenere tolleranze migliori & Tempo ciclo molto più lento\\
\midrule
Buona finitura superficiale & Scarti di materiale non recuperabili\\
\midrule
\multicolumn{2}{c}{Adatti alla lavorazione di pezzi unici}\\
\midrule
Costo macchina relativamente basso & Spesso è richiesta alta lavorazione manuale\\
\bottomrule
\end{tabularx}
\label{examp:VantSvant}
\end{example}

In industria si sta cercando di eliminare queste lavorazioni per via del tempo ciclo molto elevato.
Sebbene, in abito artigianale stiano avendo un forte sviluppo, sia in termini di automazione,
dunque di tecnologia a bordo macchina; sia di lavorazioni permesse dalle macchine.
Altro grande punto a favore di tali lavorazioni è sicuramente l'adattabilità per 
qualsiasi materiale.

In industria, vengono relegate ad operazioni di finitura del prodotto, dove con una
singola passata si cerca di completare il prodotto e metterlo sul mercato.
Per l'artigianato invece si ha un utilizzo più intensivo, dove con una passata si cerca 
di ottimizzare la quantità di materiale asportato.

Come punto di partenza è utile considerare applicazioni in cui il processo sia completamente
idealizzato.

\section{Taglio ortogonale ideale}
Alla figura \ref{fig:TaglioOrto} sono rappresentati degli esempi di taglio ortogonale.

\begin{figure}
\centering
\subfloat[][\emph{Visualizzazione del taglio ortogonale}\label{fig:TaglioOrto}]
{\includegraphics[width = 0.8\textwidth]{TaglioOrto}}\\
\subfloat[][\emph{Parametri del taglio ortogonale}\label{fig:TaglioOrtoParam}]
{\includegraphics[width=0.7\textwidth]{TaglioOrtoScheme}}
\caption{Taglio Ortogonale}\label{fig:TaglioOrto}
\end{figure}
Si dice che il taglio è ortogonale quando la lama dell'utensile è ortogonale alla direzione della velocità di taglio.
Allora si possono definire diversi parametri come riportato nella definizione \ref{def:ParamTaglioOrto}.

\begin{definition}{Parametri del taglio ortogonale}{pramTaglioOrto}
\begin{description}
\item[$\alpha$] Angolo di spoglia superiore o frontale. Da cui
	\begin{description}
	\item[Se $\alpha > 0$] allora si dice che l'utensile ha angolo acuto,
	\item[Se $\alpha < 0$] allora si dice che l'utensile ha angolo ottuso.
	\end{description}
\item[$\theta$] Angolo di spoglia inferiore
\item[$\phi$] Angolo di taglio
\item[$h$] Spessore di taglio indeformato
\item[$h_c$] Spessore del truciolo
\end{description}
\label{def:ParamTaglioOrto}
\end{definition}

Come ipotesi ideale, si considererà che tutta la deformazione del materiale avverrà
solamente sul piano di taglio.
Allora, dai parametri del taglio ortogonale si possono ottenere:
\begin{equation}
r_c = \frac{h}{h_c} = \frac{l_c}{l} := \text{Rapporto di taglio}
\end{equation}
Dove:\\
\begin{tabular}{cl}
$l_c$ & è la lunghezza del truciolo\\
$l$ & è la lunghezza del taglio\\
\end{tabular}
\\
Mentre:
\begin{equation}
F = K \cdot A
\end{equation}
ovvero, la forza necessaria a tagliare il pezzo sarà proporzionale all'area $A$ del piano di taglio 
dovuta dalla lunghezza del piano stesso per la profondità del pezzo in senso ortogonale alla 
figura \ref{fig:TaglioOrtoParam}; in più sarà proporzionale alla pressione $K$ esercitata dall'utensile sul pezzo
Inoltre, si può intuire che: per abbassare l'intensità della forza complessiva per il taglio
sarebbe opportuno aumentare $\phi$, così da limitare l'estensione del piano di taglio e di 
conseguenza la sua area.
Si può ottenere una stima dell'angolo di taglio tramite
\begin{equation}
\tan\phi = \frac{r_c \cos\alpha}{1-r_c\sin\alpha}
\end{equation}
considerando che il rapporto di taglio lo si può misurare abbastanza facilmente dato che:
lo spessore di taglio lo si decide in base alla lavorazione da effettuare, mentre lo
spessore del truciolo lo si può misurare abbastanza facilmente.
Dunque è evidente che l'angolo $\alpha$ lo si vuole molto grande, in modo da generare piani
di taglio molto piccoli: garantendo la necessità di applicare una forza minore.

Risulta utile prestare attenzione alla successiva considerazione.
Il valore di deformazione che si raggiunge con le lavorazioni per asportazione di truciolo
è nettamente più alte che non le deformazioni per lavorazioni tramite deformazione.
Giusto per dare un'idea dell'ordine di grandezza:
\begin{equation}
\dot{\gamma} = \frac{v_s}{d} = \frac{\cos\alpha}{\cos\left(\phi - \alpha\right)} \frac{v}{d} \left[\unit{\s^{-1}}\right]
\end{equation}
nella tecnica, si osservano i seguenti valori:
\begin{description}
\item[Lavorazione per deformazione] $\approx 1 \div 10 \left[\unit{\s^{-1}}\right]$
\item[Lavorazione per asportazione] $\approx 1000 \left[\unit{\s^{-1}}\right]$
\end{description}

\subsection{Fattore di attrito}
Sappiamo che il fattore di attrito ricopre importante ruolo per quanto riguarda questo tipo
di lavorazioni. Ci si deve prestare attenzione.

\begin{equation}
\mu = \frac{\tau_i}{p}
\label{eqn:FattoreAttritoIndef}
\end{equation}
Di base, questa sarebbe la definizione principale di fattore di attrito.
Si nota che passata la tensione tangenziale di Von Misess, 
l'attrito tende a calare, nonostante nell'equazione non sia previsto tale 
comportamento.
Ciò è dovuto al fatto che per calcolare il fattore di attrito si considera un materiale
indeformabile, quando nella realtà del taglio ortogonale, non può essere.
Dunque il materiale sarà \textbf{indeformabile} fino alla tensione di Von Misess, 
passata quella bisogna considerarlo \textbf{deformabile}.
Perciò l'equazione \eqref{eqn:FattoreAttritoIndef} non descrive completamente il comportamento.
Si è sviluppato un \texttt{fattore di attrito} adatto a tale fenomeno:
\begin{equation}
m = \frac{\tau_i}{K}
\label{eqn:FattoreAttritoDef}
\end{equation}
Dove $K$ è la tensione tangenziale all'interfaccia truciolo-utensile.
Nella pratica viene preferito il fattore descritto dalla \eqref{eqn:FattoreAttritoDef} 
perché più facile da calcolare. Oltre al fatto che permette di descrivere correttamente
il fenomeno dello \eng{sticking}.

\begin{definition}{Sticking}{*}
Fenomeno di aderenza tra il truciolo e l'utensile. 
\end{definition}

\subsection{Valutazione delle forze}
Per la valutazione delle forze messe in gioco per tale lavorazione si considereranno
i tre attori separatamente e man mano sovrapponendone gli effetti.

\begin{figure}
\centering
\includegraphics[width = \textwidth]{TaglioOrtoScheme}
\caption{Scomposizione delle forze per il taglio ortogonale}
\label{fig:TaglioOrtoForces}
\end{figure}

\subsubsection*{Porta utensile}
Considerando la scomposizione delle forze della figura \ref{fig:TaglioOrtoForces}.
Al porta utensile si può, operativamente, applicare una cella di carico per 
valutare la risultate applicata. Celle di carico più moderne possono eventualmente
valutare già le componenti di tale risultante.
allora possiamo ottenere i moduli delle forze:
\begin{equation}
\begin{cases}
F_c = R \cos(\alpha + \psi) &\text{Forza di taglio}\\
F_t = R \sin(\alpha + \psi) &\text{Forza di spinta dell'utensile}
\end{cases}
\end{equation}
Si possono definire ulteriori scomposizioni di forze: se prima la scomposizione
avveniva su proiezioni degli assi di riferimento rispetto al pezzo lavorato; ora si 
possono definire delle scomposizioni rispetto all'utensile.

\subsubsection*{Utensile}
Sempre in riferimento alla figura \ref{fig:TaglioOrtoForces}.
\begin{equation}
\begin{cases}
N = F_c \cos\alpha - F_t\sin\alpha\\
F = F_c \sin\alpha - F_t\cos\alpha
\end{cases}
\end{equation}
L'angolo relativo tra $R$ e $N$ viene indicato con $\beta$ e viene definito come 
\textbf{angolo d'attrito sulla faccia dell'utensile}. Vale:
\begin{equation}
\beta = \frac{F}{N} \approx \mu
\label{eqn:AngAttrito}
\end{equation}
Notare che $\beta$ è una sottospecie di coefficiente di attrito, infatti dipende
proprio da quest'ultimo.

\subsubsection*{Pezzo lavorato}
Lato pezzo lavorato si può vedere la totalità delle forze in gioco.
in particolare si devono evidenziare le forze:
\begin{description}
\item[$F_n$] risulta essere un contributo idrostatico per il materiale, dunque ne aumenta la 
duttilità. Aspetto che rende il taglio più difficoltoso. 
Ciò si risolve aumentando l'angolo del piano di taglio $\phi$ che si vedrà successivamente.
\item[$F_s$] è la vera e propria forza resistente al taglio. Si sviluppa come relazione della 
pressione di taglio resistente del materiale e l'area del piano di taglio.
\begin{equation}
F_s = k \cdot A
\end{equation}
Per rendere il taglio più agevole, si può ridurre l'area del piano di taglio nei modi che vedremo in seguito.
\end{description}

Riprendendo l'angolo del piano di taglio $\phi$, si hanno migliori condizioni lavorative
nel caso questo risulti essere molto grande.
Dalla letteratura si evidenziano le seguenti relazioni tra i vari angoli scomponenti le 
forze viste fino a prima:
\begin{subequations}
\label{eqn:Phi}
\begin{align}
\phi &= 45\unit{\degree} - \frac{1}{2}(\beta - \alpha) \label{eqn:Phi1}\\
\phi &= 45\unit{\degree} - (\beta - \alpha)\label{eqn:Phi2}
\end{align}
\end{subequations}
Sebbene indichino la relazione degli stessi parametri con due relazioni differenti \eqref{eqn:Phi}, entrambe hanno uno scopo preciso e nascono da considerazioni differenti.

\begin{description}
\item[\eqref{eqn:Phi1}] Si sviluppa partendo dalla considerazione che il sistema tenda
a consumare la minima energia.
\item[\eqref{eqn:Phi2}] Viene detta di \eng{Upper Bond}: ovvero ha l'obbiettivo di valutare
la massima forza necessaria per eseguire il taglio.
\end{description}

Per entrambi i casi risulta evidente che per aumentare $\phi$ si possono percorrere due strade:
\begin{description}
\item[$\searrow \beta$] siccome $\beta$ dipende dall'attrito tra truciolo e utensile, 
indipendentemente dallo \eng{sticking}, si può abbassare come angolo lubrificando 
l'interfaccia tra i  due.
\item[$\nearrow \alpha$] Aumentare l'angolo della faccia dell'utensile non è sempre una strada
percorribile, infatti si avrebbero utensili molto fini e fragili che tenderebbero a consumarsi
molto facilmente.
\end{description}

\begin{figure}
\centering
\includegraphics[width=\textwidth]{GraficiTaglioOrto}
\caption{Andamenti delle forze per il taglio ortogonale}
\label{fig:GarficiTaglioOrto}
\end{figure}

Nel grafico \ref{fig:GarficiTaglioOrto}.(a) è rappresentato l'andamento della forza di taglio 
$F_s$ giacente sul piano di taglio. Si osserva che tale forza aumenta all'aumentare dell'area 
del piano di taglio. 
Dunque è importante diminuire proprio quest'ultima: attraverso le strategie per 
aumentare $\phi$ viste in precedenza. Oltretutto tale andamento non è influenzato 
dall'angolo di spoglia superiore.
L'ipotesi che porta a tale costruzione sta nel fatto che il materiale è supposto non
incrudente. Infatti: raggiunto tale valore di sforzo tangenziale $\tau$ il materiale si 
rompe. Nel caso di materiale incrudente: si avrà una curva più che lineare, in quanto la 
deformazione porterà all'irrigidimento nella zona di taglio che necessiterà di maggiore
forza per avere la separazione del materiale.

Nel grafico \ref{fig:GarficiTaglioOrto}.(b) viene rappresentata la forza normale al piano di 
taglio $F_n$.
Questa dipende dall'angolo di spoglia superiore.
L'effetto di tale forza è stato già evidenziato: questa fornisce una specie di compressione 
idrostatica sul piano di taglio portando il materiale ad essere più duttile.
Ciò spesso si traduce in un truciolo continuo che però è problematico per la continuazione
della lavorazione: in quanto ingombrante e pericoloso.

\begin{figure}
\centering
\includegraphics[width=\textwidth]{Sticking}
\caption{Tipologia di forze prementi sulla faccia dell'utensile}
\label{fig:sticking}
\end{figure}

Nel grafico \ref{fig:sticking} viene riportato il dettaglio sulla faccia superiore 
dell'utensile.
Si possono vedere due sezioni distinte:
\begin{description}
\item[\eng{Sliding}] in questa sezione si presenta lo scivolamento del truciolo sulla faccia
dell'utensile. Questo porta una cera forza resistente contro l'utensile di tipo $\tau$: ovvero 
di sforzo tangenziale. Si ha uno strisciamento completamente regolato dal coefficiente di 
attrito. Tale resistenza si annullerà nel momento in cui c'è distaccamento tra truciolo e 
faccia.
\item[\eng{Sticking}] nella zona di adesione si manifesta un particolare fenomeno per cui
il materiale del lavorato si accumula sulla punta dell'utensile. Ciò permette di generare una 
zona di ristagno che favorisce la separazione del materiale tra quello tagliato e il
truciolo. Di fatto in alcune occasioni può essere benefico per il tagliente in quanto
diventa uno strato protettivo. La resistenza portata da tale fenomeno diventa di pura 
deformazione $\sigma$. 
\end{description}

Al momento non si è ancora approfondita la finalità dell'angolo di spoglia inferiore $\theta$.
Idealmente servirebbe ad evitare lo strisciamento tra utensile e lavorato.
Siccome nella realtà bisogna considerare anche il ritorno elastico, non è possibile ottenere
un perfetto distaccamento tra utensile e lavorando. Dunque un po' di strisciamento si 
verifica in qualsiasi occasione. Ciò provoca usura dell'utensile.
Tra l'altro questo tipo di usura è la più severa per l'utensile.


\section{Taglio ortogonale realistico}
Il taglio del materiale non avviene più nel piano
di taglio, bensì sulla zona di taglio detta anche
\textbf{Zona di taglio primaria} mostrato in 
figura \ref{fig:TaglioOrtoRealScheme}.

\begin{figure}
\centering
\subfloat[][\emph{Schematizzazione del taglio ortogonale realistico}\label{fig:TaglioOrtoRealScheme}]
{\includegraphics[width=0.5\textwidth]{TaglioOrtoRealScheme}}\\
\subfloat[][\emph{Tipologie di truciolo}\label{fig:Trucioli}]
{\includegraphics[width = \textwidth]{TaglioOrtoReal}}
\caption{Taglio ortogonale reale}
\label{fig:TaglioOrtoreal}
\end{figure}

Inoltre si osserva una zona deformata plasticamente, per cui si presenta un incremento della 
resistenza pari a quasi 3 volte la durezza nominale del pezzo lavorato.

Analizzando il comportamento del truciolo si possono
verificare delle situazioni, come illustrato in figura \ref{fig:Trucioli}:
\subsection{Truciolo continuo} Si verifica per velocità abbastanza basse. Come dice il nome Truciolo forma un continuo filamento di materiale.
Risulta più semplice lubrificare la faccia dell'utensile, abbassando $\beta$ come avevamo visto
in precedenza per diminuire la dimensione del piano di taglio. Dunque la zona di taglio resta più sottile
e la superficie uscirà dalla lavorazione con una 
finitura migliore.

Se da questa situazione si aumenta la velocità di
lavorazione, si va a formare una zona detta: 
\textbf{tagliente di riporto}.
Ciò è dovuto all'aumento della velocità con conseguente formazione della zona di ristagno
davanti alla faccia dell'utensile.
In particolare il tagliente di riporto:
\begin{itemize}
\item $\alpha$ aumenta per via per effetto della
zona di ristagno, per ciò si abbassa la forza di 
taglio.
\item In qualche misura protegge la faccia dell'utensile.
\item Può depositarsi sulla faccia del lavorato 
intaccandone la finitura superficiale.
\end{itemize}

Aumentando ulteriormente la velocità; la zona di taglio primario si assottiglia ulteriormente.
Comincia ad instaurarsi il fenomeno dello \eng{stiching}.
Il fenomeno si deve principalmente al riscaldamento
del materiale il quale crea adesione sulla faccia del
tagliente. Sarà, dunque, necessario aumentare la 
forza di taglio.

Si può già intuire che la velocità di taglio sarà 
strettamente legata alla temperatura generata durante la lavorazione.

Rimanendo nel tema del truciolo continuo si può verificare una situazione di truciolo continuo ma
ondulato

\subsubsection{Truciolo continuo ondulato}
È dovuto ad una variazione periodica nella forza di taglio. Possono esserci diversi motivi per cui 
si verifichi tale variazione: 
\begin{itemize}
\item Superficie rugosa, limita lo scorrimento del
tagliente.
\item Presenza di vibrazioni autoeccitate: se le vibrazioni sono dovute a forti variazioni nella superficie di taglio.
\item Distacchi del tagliente di riporto che va ad aderire alla superficie del materiale lavorato. 
Impone una maggiore forza perché l'utensile "deve"
far aderire tale riporto sulla superficie.
\item Vibrazioni forzate: generate dalla macchina 
e in genere dovute agli elementi elastici della macchina.
\item Quando si rompe il tagliente di riporto, portandosi via anche un pezzo del tagliente vero
e proprio.
\end{itemize}

Altra situazione particolare, di cui tenere nota, è la casistica in cui si forma il truciolo a dente di sega.

\subsubsection{Truciolo continuo a dente di sega}
Costituito da sezioni a diversa deformazione, la successione di zone spesse a bassa deformazione a zone fine di alta deformazione.
Le cause di tale comportamento è da imputarsi a:
\begin{itemize}
\item bassa conduttività termica del materiale,
per cui non riesce a smaltire il calore generato dalla lavorazione. 
\item Ciò spiega l'alternanza delle zone ad alta deformazione e bassa deformazione: le sezioni a bassa deformazione sono caratterizzate da un'alta temperatura relativa per cui l'utensile taglia
più facilmente il materiale. Una volta asportata una sezione di materiale, viene automaticamente smaltito
del calore "bloccato" nel materiale. Dunque la sezione successiva trova un calore minore, per cui il 
tagliente fa più fatica a tagliare e deforma di più 
il materiale. Per questo motivo le sezioni ad alta 
deformazione sono più fine.
\end{itemize}
Ultima casistica di truciolo, è quella del truciolo 
discontinuo.

\subsection{Truciolo discontinuo}
Si verifica in situazioni di velocità estremamente
basse su macchine scadenti a bassa rigidezza.
Altre situazioni per cui si va ad avere truciolo 
discontinuo sono:
\begin{itemize}
\item Per materiali che contengono inclusioni e fasi
secondarie che alzano le tensioni. Ciò provocano delle continue rotture di truciolo.
\item Situazioni di stick-slip.
\end{itemize}

Sebbene il truciolo continuo sia ricercato per favorire una finitura superficiale migliore, spesso
non è possibile tenere trucioli particolarmente lunghi. Durante la lavorazione un truciolo lungo
può essere d'intralcio all'operatore e alla macchina.

Allora si ha la necessità di rompere il truciolo.

\begin{wrapfloat}{figure}{O}{0pt}
\includegraphics[width = 0.3\textwidth]{Rompitruciolo}
\caption{Esempi di rompi-truciolo}
\label{fig:Rompitruciolo}
\end{wrapfloat}


\subsection{La rottura del truciolo}
È particolarmente facile coi materiali fragili, più
complicato per quei materiali duttili che tendono
a formare dei trucioli continui.
Allora si ricorre a opportuni \textbf{rompi-truciolo},
inserti che possono essere fissati all'utensile o già
disegnati su esso, per favorire la rottura del 
truciolo ponendo ulteriore deformazione al materiale
asportato.
I rompi-truciolo hanno uno specifico valore efficace
di rottura del truciolo dipendente dalla profondità 
dello spessore indeformato: questa è una 
caratteristica della forma e dimensione geometrica 
del rompi-truciolo. In figura \ref{fig:Rompitruciolo} 
un dettaglio.

\section{Taglio obliquo}
Viene sfruttato maggiormente dalle aziende.
Garantisce una lavorazione che necessita di maggiore
forza per il taglio, ma la vita del utensile
è più lunga.

A differenza del taglio ortogonale dove si veniva
a formare un truciolo continuo a forma di spirale.
Nel taglio obliquo si forma un truciolo a forma 
elicoidale, che ha una propria tendenza geometrica
a spostarsi via dal punto di lavorazione.
Si evita la necessità di rompere il truciolo, anche 
se è una pratica che comunque viene mantenuta.
È opportuno definire un nuovo angolo di spoglia 
superiore.
\begin{align}
\alpha_e &:= \text{Angolo di spoglia efficace}\\
\alpha_n &:= \text{Angolo di spoglia normale}\\
&\text{In generale: }\alpha_e > \alpha_n
\end{align}
Vale che più il tagliente è inclinato, più aumenta $\alpha_e$ 
e di conseguenza $F_c$ diminuisce.

Nel caso in cui il tagliente non sia più grande del
pezzo in lavorazione, lo spessore del truciolo 
indeformato non è più sufficiente a descrivere i 
fenomeni legati a questo. Dunque è necessario parlare
di \textbf{avanzamento} $f$. Inoltre si introduce 
la \textbf{profondità di passaggio} $w$.

\begin{figure}
\centering
\subfloat[][\emph{Schema del taglio obliquo}\label{fig:TaglioObliquoScheme}]
{\includegraphics[width = \textwidth]{TaglioObliquo}}\\
\subfloat[]
[\emph{Parametri del taglio ortogonale per confronto}\label{fig:TaglioObliquoParam1}]
{\includegraphics[width = 0.4\textwidth]{TaglioObliquoParam1}}\quad
\subfloat[][\emph{Ulteriori parametri del taglio obliquo}\label{fig:TaglioObliquoParam2}]
{\includegraphics[width = 0.4\textwidth]{TaglioObliquoParam2}}
\caption{Il taglio obliquo}
\label{fig:TaglioObliquo}
\end{figure}

\subsection{Tornitura}
Un esempio di tornitura è riportato alla figura \ref{fig:TaglioObliquoParam2}.
\begin{wrapfloat}{figure}{O}{0pt}
\includegraphics[width = 0.5\textwidth]{Utensile}
\caption{Esempio di utensile per tornitura}
\label{fig:UtensileTornitua}
\end{wrapfloat}

\begin{itemize}
\item utensile con tagliente inclinato
\item $f$ è ortogonale al tagliente e rappresenta 
l'avanzamento;
\item $w$ è la profondità di passata
\item $h$ è lo spessore di truciolo indeformato.
\end{itemize}

Con l'utensile inclinato si ha una maggiore forza 
necessaria al taglio, però si aumenta 
considerevolmente la durata dell'utensile.
Indubbiamente la caratterizzazione del tagliente 
risulta più complicata. Si ha l'ulteriore vantaggio
di avere un utensile che può presentare più 
taglienti.

\subsection{Calcolo della forza ed energia}
Siccome per via del tipo di lavorazione non siamo in grado di sfruttare
la deformazione per flusso plastico a causa della velocità decisamente 
superiore.
È decisamente importante sapere quanta energia o potenza è necessaria 
per effettuare la lavorazione. Perché tale potenza sarà quella che il
motore della macchina deve fornire all'utensile.

A differenza della lavorazione per deformazione plastica in cui 
la "pressa" deve fornire una certa pressione per permettere la 
deformazione. Nell'asportazione si possono cambiare alcuni
parametri in base anche punto della lavorazione si è.
Continuando a lavorare il pezzo.

Resta evidente come la parametrizzazione della lavorazione sia
definitivamente importante: in modo da garantire la continuazione
della lavorazione, cambiando alcune condizioni senza interromperla.
Anche per il fatto che fermando la lavorazione peggiora la finitura.

Nella definizione dell'energia necessaria si sa che si va a commettere
un'errore di circa il $20\%$.
Di fatto sarebbe necessario conoscere $\beta$, ma non è facile da valutare.
Allora si può valutare la \textbf{pressione di taglio}.

\begin{equation}
p_c = \frac{\overbrace{F_c}^{\text{Componente parallela alla velocità di taglio}}}{h \cdot w}
\end{equation} 
Per ottenere una forma di energia basta integrare la pressione
o forza per la lunghezza del lavoro eseguito ovvero:
\begin{equation}
l := \text{lunghezza di taglio}\\
\end{equation}

\begin{definition}{Pressione ed energia di taglio specifica}{PEspec}
\begin{subequations}
\begin{align}
p_c &= \frac{F_c}{h \cdot w} \\
E_1 &= p_c \cdot \frac{l}{l} \:[\unit{\W\s/\m^3}]
\end{align}
\end{subequations}
Dove $p_c$ è la pressione specifica di taglio e $E_1$ è l'energia specifica.
\end{definition}

\begin{definition}{Fattore di rimozione}{FattRim}
Si può definire anche il Fattore di rimozione del materiale:
\begin{equation}
k_1 = \frac{1}{E_1}
\end{equation}
dove $k_1$ rappresenta la quantità di materiale asportato da una macchina
a motore a potenza unitaria.
\end{definition}

I 3 parametri non sono costanti per un materiale perché dipendono anche dai
parametri di processo quali spessore di truciolo indeformato, angolo di spoglia e
velocità di taglio.
L'energia spesa nella zona di taglio primaria e la quantità di materiale rimosso
sono proporzionali allo spessore di truciolo indeformato.
\begin{equation}
E = k \cdot h \cdot p \cdot l
\end{equation}

Nella zona di taglio primaria la pressione specifica, l'energia specifica e il fattore di
rimozione del materiale sono costanti del materiale.
L'energia consumata sul dorso dell'utensile è indipendente dallo spessore di
truciolo indeformato, possiamo ritenerla quasi costante.
Quindi vale:
\begin{equation}
\frac{E_d}{h \cdot p \cdot l} \neq cost.
\end{equation}

A questo punto si è definita una relazione in cui si può già definire la forza necessaria.
C'è da tenere in considerazione l'usura dell'utensile: in quanto quest'ultima tende
a richiedere una maggiore pressione da parte del tagliente per tagliare.
Siccome l'usura del tagliente è un argomento che richiede la considerazione di diversi
parametri e che tali non possono sempre essere determinatiti così facilmente.
Semplicemente si aggiunge un $30\%$ in più all'energia necessaria.

Ulteriori parametri derivati dai precedenti sono:
\begin{definition}{Ulteriori Parametri}{UltParam}
Sono delle derivazioni dei parametri visti precedentemente.
\begin{equation}
E = E_1 \left(\frac{h}{h_{ref}}\right)^{-a} = E_1 \cdot h^{-a}
\end{equation}
Dove $E$ è l'energia di lavorazione, $h_{ref} = 1\unit{\mm}$ è lo spessore indeformato di 
riferimento e $a \approx 0.3$.
Da cui si può definire la potenza necessaria per la lavorazione:
\begin{equation}
Power(W) = \frac{E \cdot V_t}{\eta}
\end{equation}
Dove $V_t$ è la velocità di taglio, $\eta$ il rendimento della macchina.
Da cui si può ottenere la forza necessaria per il taglio:
\begin{equation}
F_c = \frac{Power(W)}{v}
\end{equation}
\end{definition}

\subsection{Temperature}
Siccome la lavorazione prevede di "muovere" del materiale dentro se stesso
questo provoca un accumulo di energia danti alla faccia dell'utensile.
Tale energia rimane nella maggior parte nel truciolo.

\begin{wrapfloat}{figure}{I}{0pt}
\includegraphics[width = 0.4\textwidth]{Temperature}
\caption{Analisi termica dell'asportazione di truciolo}
\label{fig:Temp}
\end{wrapfloat}

Come si vede dalla figura \ref{fig:Temp}, le temperature raggiunte non sono da 
trascurare: si può arrivare a $\approx 1000\unit{\celsius}$ sulla faccia dell'utensile.
Sappiamo che con l'aumento della temperatura, aumentano il numero di dislocazioni
interne al materiale che diventa più duttile dunque più facile da lavorare.

La maggior parte dell'energia resta interna al truciolo che, portando con se del
calore, abbassa la temperatura del lavorato. Resta il calore trasferito 
all'utensile, il quale tende ad aumentare la temperatura considerevolmente.
Infatti, come si vede proprio dalla figura \ref{fig:Temp}, è proprio
l'utensile a mostrare le temperature più alte. Ciò impone che il
materiale dell'utensile debba essere di un materiale alto resistente a caldo.

\begin{figure}
\centering
\subfloat[][\emph{Schema delle isoterme sulla faccia dell'utensile}\label{fig:TempUtensile}]
{\includegraphics[width = 0.4\textwidth]{TempScheme}}\quad
\subfloat[][\emph{Relazione tra velocità di lavorazione e temperatura}
\label{fig:TempVel}]
{\includegraphics[width = 0.4\textwidth]{TempVel}}
\caption{Temperature durante le lavorazioni ad asportazione}
\label{fig:TempLav}
\end{figure}

Come fatto in precedenza, si è cercato di relazionare la temperatura con la
velocità di lavorazione: legame che fisicamente è noto ma non così facile 
da porre analiticamente.

\begin{equation}
T_t = E \left( \frac{v \cdot h}{k \cdot \rho \cdot c}\right)^{1/2}
\end{equation}
Dove:\\
\begin{tabularx}{\textwidth}{cX}
$E_1$ & Energia specifica\\
$v$ & Velocità di taglio\\
$h$ & Spessore di truciolo indeformato\\
$k$ & Conduttività termica\\
$\rho$ & Densità\\
$c$ & Calore specifico 
\end{tabularx}\\
Se ne ottiene una stima della temperatura media sulla faccia dell'utensile.
Come spesso succede nei sistemi fisici reali, tutta l'energia donata dalla
macchina al materiale viene trasformata in calore.
Indubbiamente durante le lavorazioni non si deve superare la temperatura di fusione
del materiale lavorato: altrimenti si va ad ottenere una lavorazione molto grossolana di
cui non si riesce a controllarne i parametri.
I materiali a bassa fusione risultano più facili da lavorare.
Inoltre, per lavorazioni di finitura particolarmente restrittive, bisogna considerare
le variazioni dimensionali dovute alla dilatazione termica.

\subsection{Fluidi da taglio}