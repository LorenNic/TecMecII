\chapter{Lavorazioni non convenzionali}\label{chp:NonConvenzionale}
Sono caratterizzate dal fatto che non usano l'energia meccanica per modellare il materiale.
Dato questo, la durezza del materiale non costituisce un limite a tale problema. Per cui ci sono diverse possibilità anche per materiali non normalmente lavorabili.

\missingfigure{Grafico lavorazioni non convenzionali}

\begin{description}
\item[Attacco chimico] Si vuole rimuovere del materiale tramite azione chimica. Allora il problema non è più come togliere il materiale, piuttosto controllarla.
\item[Elettroerosione] Si sfrutta il movimento degli atomi per separarli dal lavorato.
\item[Scarica elettrica] Si sfrutta un'arco elettrico per separare il materiale. Se ne era già parlato per la generazione delle polveri per la sinterizzazione
\item[lavorazioni ad alto contenuto energetico] Si parla di trasmettere elettroni ad alta energia per la separazione del materiale.
\item[Altre tipologie] in cui si includono il taglio ad acqua (anche se potrebbe rientrare tra le tecniche abrasive), lavorazioni ad ultrasuoni ecc\dots
\end{description}