\chapter{Variabili di processo per lavorazioni in asportazione di truciolo}
\label{chp:VarProcAsportazione}
GLi aspetti principali da considerare per tali lavorazioni sono:
\begin{itemize}
\item \todo{\\Inserire aspetti da considerare}
\end{itemize}

La prima decisione necessaria è la sequenza delle operazioni da eseguire.
Per ogni lavorazione bisogna scegliere velocità di taglio e avanzamento in funzione dei materiali del pezzo e utensile. Anche la fase della lavorazione 
va tenuta in conto, sgrossatura e finitura hanno due obbiettivi diversi 
e necessitano considerazioni diverse.

La finitura superficiale e le tolleranze dimensionali richieste vengono ottenute eseguendo un taglio di finitura con valori piccoli di avanzamento e profondità di passata.
Per cui in caso di oggetti \eng{near net-shape}, che possono presentare degli ossidi per via di lavorazioni a caldo precedenti, deve tenere in considerazione che un'operazione di finitura andrà a consumare maggiormente l'utensile: sapendo che il cambio utensile può essere svantaggioso in termini di finitura superficiale.

In una piccola officina il tempo non è un fattore eccessivamente importante.
Spesso ci si basa sull'esperienza dell'operatore per gestire la lavorazione
Diverso è il discorso per un'azienda ad alta produttività, in cui per tenere
basso il tempo ciclo di produzione è critico.
In questo caso, il \eng{set-up} può essere particolarmente problematico, 
allora alcuni enti hanno costituito col tempo, dei manuali da cui prendere
spunto per realizzare una lavorazione abbastanza rapidamente e senza eccessivo consumo dell'utensile. 

Ricordiamo che l'usura dell'utensile è determinata dalla temperatura raggiunta dall'utensile.
la temperatura, a sua volta, dipende dalla velocità di taglio.
\missingfigure{Grafico relazione durezza-temperatura-velocità}
\todo{\\Sulla figura inserire anche il cartiglio per via delle descrizione delle diverse curve}
\missingfigure{Grafico relazione durezza-temperatura-velocità materiali non ferrosi}

I grafici precedenti \todo{\\riferimenti figure} aiutano a scegliere i principali 3 parametri per le operazioni di sgrossatura.
Gli avanzamenti vanno scelti in funzione delle durezze dei materiali in cui si va a lavorare.
In generale i grafici grafici danno un'indicazione generica, un punto di partenza prudente, per la lavorazione. Le stime si basano sul portare la durata dell'utensile accettabile per circa 1-2 ore.
Tramite prove tecnologiche, si può accelerare la lavorazione se l'utensile arriva ad un tempo superiore alle 2 ore.
I riferimenti sono per un'operazione di tornitura. I manuali danno, in genere, i parametri di riferimento per altre operazioni come: fresatura, piallatura ecc\dots
\missingfigure{Costruire tabella coi riferimenti oppure inserire la figura}
$Z_p$ e $Z_f$ sono dei coefficienti da moltiplicare ai riferimenti ottenuti dalla tornitura alle altre lavorazioni.\todo{\\Vedi la descrizione della tabella}

La foratura con punta da trapano viene trattata a parte per via della sua natura. in generale si hanno dei valori piuttosto simbolici tipo $v = 0.5v_s$ per il ferro e $v = 0.6v_s$ per materiali non ferrosi \todo{\\verifica}.
Anche la brocciatura ha dei parametri diversificati per cui viene descritta a parte.
Per esempio anche la profondità del foro gioca un ruolo importane nel controllo della temperatura.

Altri miglioramenti in termini di velocità si differenziano in base al materiale dell'utensile: per utensili ceramici si possono quasi raddoppiare le velocità di lavorazione (ovviamente da vedere nel caso specifico).

Non bisogna dimenticare anche altri aspetti tipo: rigidezza della macchina, il pezzo da ottenere, elementi di fissaggio, ecc\dots
Una macchina di alto livello riesce ad assorbire le vibrazioni non cedendole all'utensile: ottenendo un controllo migliore sulla lavorazione.

\subsection{Durata lavorazione e potenza macchina}
Una volta scelti i parametri di velocità e avanzamento di solito si stimano:
\begin{itemize}
\item la quantità di volume da rimuovere.
\item la velocità di rimozione del truciolo
\item Dai parametri precedenti si può stimare la durata della lavorazione come $t_c = \frac{\overbrace{v_c}^{\text{velocità di taglio}}}{\underbrace{V_t}_{\text{volume rimosso}}}$
\item Si stimano energia specifica e successivamente la potenza richiesta dalla macchina.
\end{itemize}
Si ricorda che la velocità di asportazione è data dalla velocità di taglio per la sezione trasversale del truciolo: spessore indeformato e profondità di passata.

\subsubsection{Esempio applicativo}
Si vuole allargare un foro in un getto in acciaio con un utensile in carburo. Il diametro iniziale è $D_i = 130\unit{\mm}$ a $D_f = 138\unit{mm}$. Si vuole suggerire velocità di taglio e avanzamento, profondità e potenza necessaria.

\subsubsection*{Soluzione}
Da manuale si ottengono: 
\begin{itemize}
\item $UTS$ del materiale: in questo caso $UTS = 485\unit{\MPa}$.
\item da cui si ricava $HB = 3*485*9.8=150\unit{\kg/\mm^2}$.
\end{itemize}
Allora possiamo stimare;
\begin{itemize}
\item Profondità di passata: $w = (138-130)/2 = 4\unit{\mm}$,
\item Dal grafico otteniamo: $v_s = 1.8\unit{\m/\s}$ che può essere aumentata del 20\% per via dell'utensile usa-e-getta al carburo da cui: $v_s = 1.8 * 1.2 = 2.16\unit{\m/\s}$
\end{itemize}
Siccome stiamo trattando una tornitura di interni:
\begin{itemize}
\item avanzamento: $f_s = 0.5\unit{\mm}$
\item siccome $Z_f = 1$ allora $f = 0.5\unit{\mm}$
\item Area trasversale truciolo: $A = f*w = 0.5*4 = 2\unit{\mm^2}$
\item Volume asportato: $V_t = A * v_s = 4320\unit{\mm^3/\s}$
\end{itemize}
Per calcolare la potenza:
\begin{itemize}
\item Dalla tabella: $E_1 = 2.1\unit{\W\s/\mm^2}$
\item Energia spesa: $E = E_1 * h^{-a}= 2.59\unit{\W\s/\mm^2}$
\item Da cui la potenza: $P = \frac{2.59 * 4320}{0.7} = 16\unit{\kW}$
\item la forza di taglio: $P_c = \frac{15956}{2.16} = 7.4\unit{\kN}$
\end{itemize}

%%%%%%%%%%%%%%%%%%%%%%%%%%%%%%%%%%%%%%%%%%%%%%%%%%%%%%%%%%%%%%%%%%%%%
% Macchine Utensili											       %
%%%%%%%%%%%%%%%%%%%%%%%%%%%%%%%%%%%%%%%%%%%%%%%%%%%%%%%%%%%%%%%%%%%%%
\chapter{Macchine Utensili}\label{chp:MacchineUtensili}
Spesso il cliente chiede delle macchine con delle specifiche particolari
in base alle lavorazioni che necessita.
In generale sono le dimensioni che differenziano le macchine.
Può capitare che ci siano delle necessità particolari per delle operazioni 
molto particolari.

Un caso particolare è rappresentato dalle macchine esapodi: presentano sei colonne che possono allungarsi e accorciarsi. Il vantaggio è che data la geometria impedisce alle colonne di flettere ma solo di porsi in trazione e compressione.
Ciò dovrebbe garantire altissima precisione e accuratezza nella lavorazione ma non hanno avuto grande successo per via del grande ingombro rispetto al volume lavorabile. Dunque poco adatte per una lavorazione industriale.

Si era già accennato lo sviluppo delle macchine a controllo numerico.
La diffusione di tali macchine è dovuta all'integrazione tra macchina e 
processore.

Ulteriore problematica ed eventuale miglioramento delle macchine sta nella capacità di assorbire le vibrazioni.

Passando ad un altro aspetto: ovvero la stabilità termica.
In generale una macchina provoca calore nella lavorazione. Dunque è necessario considerare l'eventuale dilatazione termica dovuta all'incremento della temperatura.
Allora sono state sviluppate delle tecniche di controllo adattativo: ovvero le macchine riescono a compensare delle variazioni durante la lavorazione, proprio come la dilatazione termica. 
Per esempio può essere rilevata la variazione dimensionale di un pezzo proprio per dilatazione termica, allora si può adattare la profondità dell'utensile per compensare tale variazione. 
Per questo scopo devono essere montati dei sensori abbordo macchina.
Eventualmente è necessario scegliere dei motori della macchina che possano compensare eventuali errori nel posizionamento come i motori a passo.

